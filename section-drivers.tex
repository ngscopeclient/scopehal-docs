\section{Oscilloscope Drivers}
\label{sec:drivers}

\subsection{Agilent}

Agilent devices support a similar similar SCPI command set across most device families. It is usually made available over Ethernet, USB, and GPIB. Only Ethernet is currently supported..

Please see the table below for details of current hardware support:

\begin{tabularx}{16cm}{llX}
\thickhline
\textbf{Device Family} & \textbf{Notes} \\
\thickhline
DSO5000 series & Not recently tested, but should work.\\
\thickhline
DSO6000 \& MSO6000 series & Working. No support for digital channels yet.\\
\thickhline
DSO7000 \& MSO7000 series & Untested, but should work. No support for digital channels yet.\\
\thickhline
\end{tabularx}

\subsubsection{agilent\_lan}

This driver uses SCPI over TCP, on port 5025.

It takes two arguments: hostname/IP and port number. If omitted, port number is assumed to be 5025.

Example:
\begin{lstlisting}[language=sh]
./glscopeclient --debug myscope:agilent_lan:192.168.1.1:5025
\end{lstlisting}

This driver has been tested on an MSO6034A.

\subsection{Antikernel Labs}

Under-development internal logic analyzer analyzer core for FPGA design debug. The ILA uses a UART interface to a host
system. Since there's no UART support in scopehal yet, socat must be used to bridge the UART to a TCP socket
(SCPISocketTransport)

\subsubsection{akila\_lan}

This driver uses a raw binary protocol over TCP.

It takes two arguments: hostname/IP and port number. If omitted, port number is assumed to be 5555.

\subsection{Enjoy Digital}
TODO (scopehal:79)

\subsection{Hantek}
TODO (scopehal:26)

\subsection{Keysight}
TODO

\subsection{Pico Technologies}
TODO (scopehal:15)

\subsection{Rigol}
TODO (scopehal:12)

\subsection{Rohde \& Schwarz}
TODO (scopehal:59)

\subsection{Saleae}
TODO (scopehal:16)

\subsection{Siglent}

Many recent Siglent oscilloscopes are developed in partnership with Teledyne LeCroy (Siglent-designed hardware running
Teledyne LeCroy firmware) and are sold under both brands. As a result, the Teledyne LeCroy compatible drivers work with
many Siglent devices.\\

This driver may be selected by either requesting \texttt{siglent\_lan} or \texttt{lecroy\_lan} in glscopelient command
string. This driver will use port 5025 by default. \\

\begin{tabularx}{16cm}{llX}
\thickhline
\textbf{Device Family} & \textbf{Notes} \\
\thickhline
SDS-1000X-E series & Base functionality present. Tested on SDS-1204X-E. \\
\thickhline
\end{tabularx}

\subsection{Teledyne LeCroy / LeCroy}

There are currently two drivers for Teledyne LeCroy (and older LeCroy) devices. While all Teledyne LeCroy / LeCroy
devices use almost identical SCPI command sets, Windows based devices running XStream or MAUI use a custom framing
protocol around the SCPI data while the lower end RTOS based devices use raw SCPI over TCP.

Please see the table below for details on which driver to use with  your hardware.

\begin{tabularx}{16cm}{llX}
\thickhline
\textbf{Device Family} & \textbf{Driver} & \textbf{Notes} \\
\thickhline
DDA & lecroy\_vicp & \\
\thickhline
HDO & lecroy\_vicp & \\
\thickhline
LabMaster & lecroy\_vicp & Untested, but should work\\
\thickhline
MDA & lecroy\_vicp &  Untested, but should work\\
\thickhline
SDA & lecroy\_vicp &  Untested, but should work\\
\thickhline
T3DSO & lecroy\_lan & TODO \\
\thickhline
WaveAce & lecroy\_lan & TODO \\
\thickhline
WaveJet & lecroy\_lan & TODO \\
\thickhline
WaveMaster & lecroy\_vicp & Untested, but should work \\
\thickhline
WaveRunner & lecroy\_vicp &  \\
\thickhline
WaveSurfer & lecroy\_vicp &  \\
\thickhline
\end{tabularx}

\subsubsection{lecroy\_lan}

TODO: finish this one

\subsubsection{lecroy\_vicp}

This driver uses SCPI over the Teledyne LeCroy Versatile Instrument Control Protocol (VICP), which runs over TCP port
1861.

It takes two arguments: hostname/IP and port number. If omitted, port number is assumed to be 1861.

Example:
\begin{lstlisting}[language=sh]
./glscopeclient --debug myscope:lecroy_vicp:192.168.1.1:1861
\end{lstlisting}

This driver has been tested on WaveSurfer 3034, WaveRunner 8104, and HDO9204 devices. It should be compatible with any
Teledyne LeCroy oscilloscope running Windows 7 or newer and the MAUI software.

While the driver is expected to be compatible with older Windows XP based LeCroy devices running XStream, as of this
writing no testing has been conducted. Reports of tests, successful or otherwise, will be appreciated!

\subsection{Tektronix}
TODO (scopehal:73, scopehal:13)

\subsection{Xilinx}
TODO (scopehal:40)
