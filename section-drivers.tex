\section{Oscilloscope Drivers}
\label{sec:drivers}

\subsection{Agilent}

Agilent devices support a similar similar SCPI command set across most device families.

Please see the table below for details of current hardware support:

\begin{tabularx}{16cm}{llX}
\thickhline
\textbf{Device Family} & \textbf{Driver} & \textbf{Notes} \\
\thickhline
DSO5000 series & agilent & Not recently tested, but should work.\\
\thickhline
DSO6000 \& MSO6000 series & agilent &  Working. No support for digital channels yet.\\
\thickhline
DSO7000 \& MSO7000 series & agilent &  Untested, but should work. No support for digital channels yet.\\
\thickhline
\end{tabularx}

\subsubsection{agilent}

Example:
\begin{lstlisting}[language=sh]
./glscopeclient --debug myscope:agilent:lan:192.168.1.1:5025
\end{lstlisting}

This driver has been tested on an MSO6034A.

\subsection{Antikernel Labs}

So far only the ILA IP is supported.

\begin{tabularx}{16cm}{llX}
\thickhline
\textbf{Device Family} & \textbf{Driver} & \textbf{Notes} \\
\thickhline
Internal Logic Analyzer & akila & \\
\thickhline
\end{tabularx}

\subsubsection{akila}

This driver uses a raw binary protocol, not SCPI.

Under-development internal logic analyzer analyzer core for FPGA design debug. The ILA uses a UART interface to a host
system. Since there's no UART support in scopehal yet, socat must be used to bridge the UART to a TCP socket using
the ``lan" transport.

\subsection{Enjoy Digital}
TODO (scopehal:79)

\subsection{Hantek}
TODO (scopehal:26)

\subsection{Keysight}
TODO

\subsection{Pico Technologies}
TODO (scopehal:15)

\subsection{Rigol}
TODO (scopehal:12)

\subsection{Rohde \& Schwarz}
TODO (scopehal:59)

\subsection{Saleae}
TODO (scopehal:16)

\subsection{Siglent}

Many recent Siglent oscilloscopes are developed in partnership with Teledyne LeCroy (Siglent-designed hardware running
Teledyne LeCroy firmware) and are sold under both brands. As a result, there is some crossover in driver support. \\

\begin{tabularx}{16cm}{llX}
\thickhline
\textbf{Device Family} & \textbf{Driver} & \textbf{Notes} \\
\thickhline
SDS-1000X-E series & siglent & Base functionality present.\\
\thickhline
\end{tabularx}

Tested on SDS-1204X-E.

\subsection{Teledyne LeCroy / LeCroy}

Teledyne LeCroy (and older LeCroy) devices use the same driver, but two different transports for LAN connections.

While all Teledyne LeCroy / LeCroy devices use almost identical SCPI command sets, Windows based devices running
XStream or MAUI use a custom framing protocol (``vicp") around the SCPI data while the lower end RTOS based devices use
raw SCPI over TCP (``lan"). Some of these devices also require use of the Siglent driver as they are Siglent OEM
designs rebranded by Teledyne LeCroy and have some quirks in the firmware which require workarounds.

Please see the table below for details on which configuration to use with  your hardware.

\begin{tabularx}{16cm}{lllX}
\thickhline
\textbf{Device Family} & \textbf{Driver} & \textbf{Transport} & \textbf{Notes} \\
\thickhline
DDA & lecroy & vicp & \\
\thickhline
HDO & lecroy & vicp & \\
\thickhline
LabMaster & lecroy & vicp & Untested, but should work\\
\thickhline
MDA & lecroy & vicp & Untested, but should work\\
\thickhline
SDA & lecroy & vicp & Untested, but should work\\
\thickhline
T3DSO & siglent & lan & Untested, but should work\\
\thickhline
WaveAce & siglent & lan & Untested, but should work \\
\thickhline
WaveJet & siglent & lan & Untested, but should work \\
\thickhline
WaveMaster & lecroy & vicp & Untested, but should work \\
\thickhline
WaveRunner & lecroy & vicp & \\
\thickhline
WaveSurfer & lecroy & vicp & \\
\thickhline
\end{tabularx}

\subsubsection{lecroy}

This is the primary driver for MAUI based Teledyne LeCroy / LeCroy devices.

Example:
\begin{lstlisting}[language=sh]
./glscopeclient --debug myscope:lecroy:vicp:192.168.1.1:1861
\end{lstlisting}

This driver has been tested on a wide range of Teledyne LeCroy / LeCroy hardware including DDA 5005, DDA 5005A,
WaveSurfer 3034, WaveRunner 8104, and HDO9204. It should be compatible with any Teledyne LeCroy or LeCroy oscilloscope
running Windows XP or newer and the MAUI or XStream software.d!

\subsection{Tektronix}
TODO (scopehal:73, scopehal:13)

\subsection{Xilinx}
TODO (scopehal:40)
