\chapter{Tutorials}

This section contains step-by-step examples to help you familiar yourself with the basics of using ngscopeclient.

Many of them can be done offline using the built-in demo oscilloscope or saved example waveform datasets - no lab
access required!

\section{The Basics}

\begin{itemize}
\item \textbf{Lab requirements:} None, can be performed offline
\item \textbf{Learning goals:} Navigating the ngscopeclient UI and performing common operations
\end{itemize}

\subsection{Connecting to an Oscilloscope}

We need to be connected to an instrument to do much of anything useful. Let's use the built-in demo scope for that.

\begin{enumerate}
\item Start with a new, empty ngscopeclient session
\item Select \menustyle{Add / Oscilloscope / Connect...} from the top menu.
\item Select the ``demo" driver and ``null" transport. Leave the path blank, since the demo scope doesn't need any
connection information.
\item Give the demo scope a nickname of your choice. This will be used to disambiguate scope channels and properties
dialogs if you have more than one instrument connected, as well as to let you reconnect to the instrument quickly in
the future.

\begin{figure}[h]
\centering
\bigimage{ng-images/tutorials/basics/connect-example.png}
\caption{Connection dialog}
\label{basic-connect}
\end{figure}

\item Click the ``add" button. You should be presented with an oscilloscope view showing four empty channels stacked on
top of each other.

\begin{figure}[h]
\centering
\bigimage{ng-images/tutorials/basics/connected.png}
\caption{Application window after connecting to demo scope}
\label{basic-connected}
\end{figure}

\end{enumerate}

\FloatBarrier

\subsection{Acquiring Waveforms}

Looks pretty boring! Let's grab some waveforms so we have something to look at.

Note: The current demo scope is a simplistic instrument that doesn't implement realistic trigger semantics, so most of
the usual trigger settings you might expect from real scopes (adjusting trigger level, horizontal position, selecting
type of edge or condition) aren't available. It will always trigger immediately when armed and return waveforms at the
same horizontal position.

\begin{enumerate}
\item Press the ``single trigger" button (second from left on the toolbar). You should see waveforms appear in each
channel.

\begin{figure}[h]
\centering
\bigimage{ng-images/tutorials/basics/first-waveform.png}
\caption{Our first waveforms}
\label{basic-first}
\end{figure}

\item Press the ``normal trigger" button (leftmost on the toolbar). You should see the waveform display begin updating
live.

\item Press the ``stop" button (fourth from left on the toolbar) to stop acquiring waveforms.

\end{enumerate}

\subsection{Navigating the Y Axis}

All of the waveforms in the demo scope are centered around zero volts and just the right amplitude to fill the view.
But in real life we're usually not that lucky. Let's try moving one of the waveforms around.

\begin{enumerate}
\item Move the mouse over the Y axis at the right side of one of the plots.
\item Click and drag with the left button to move the waveform vertically (adjusting frontend offset).
\item Scroll with the mouse wheel to scale the waveform vertically (adjusting frontend gain).

\begin{figure}[h]
\centering
\bigimage{ng-images/tutorials/basics/moved-waveform.png}
\caption{Demo channel after making some gain and offset tweaks}
\label{basic-moved}
\end{figure}

\subsection{Navigating the X Axis}

The timebase in ngscopeclient is decoupled from the viewport, so you can zoom and pan arbitrarily in the X axis without
changing timebase settings.

Although we've only been looking at a $25 ns$ wide window of the waveform so far, the default settings for the demo scope
are 100K points at 50 Gsps ($2 \mu s$ record length). Let's explore the rest of the waveform.

\begin{enumerate}
\item Move the mouse over the main plot area and scroll with the mouse wheel to zoom in or out, centering at the mouse
cursor position.
\item Move the mouse over the timeline at the top of the viewport and drag with the left button to move the waveform
side to side without changing zoom.

\begin{figure}[h]
\centering
\bigimage{ng-images/tutorials/basics/zoomed-waveform.png}
\caption{Demo session after zooming out to show entire waveform}
\label{basic-zoomed}
\end{figure}

\end{enumerate}

\end{enumerate}

